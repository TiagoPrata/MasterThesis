\chapter{Simulações}
\label{ch:simulacoes}

Para a aplicação dos controladores \acrshort{mpc} criados nos capítulos anteriores, foi desenvolvido um
ambiente \textit{Simulink} (\cref{fig:simulinkusingmpc}) onde cada um dos diferentes controladores obtidos
foram conectados ao \acrshort{tclabsp} para realizar seu controle.

Todos\footnote{
    O \acrshort{matlab} não conseguiu criar um controlador baseado no modelo experimental Box-Jenkins
    pois este modelo possui pólos discretizados próximos de $z=0$.
}  os diferentes controladores foram submetidos aos mesmos sinais de referência (\textit{set-point})
e ao mesmo tempo de experimento. Os sinais de referência utilizados foram sinais de amplitude randômica
e com duração de duas vezes o tempo de acomodação da planta em malha aberta, ou seja, $2$ x $T_s = 2$ x $383s = 766s$.
A duração do experimento foi determinada de forma que cinco sinais de entrada distintos pudessem ser
aplicados ao sistema, sendo assim o tempo dos experimentos foi de $5$ x $766s = 3830s$.
Cada um dos experimentos foi repetido ao menos 3 vezes e ao final foi calculada uma média de todos os 
sinais coletados para atenuar possíveis erros pontuais de cada experimento.

\begin{figure}[!h]
	\caption{Ambiente Simulink para implementação dos controladores MPC}
	\begin{center}
		\includegraphics[width=1.00\textwidth]{./5_images/SimulinkUsingMPC.png} 
		\label{fig:simulinkusingmpc}
	\end{center}
	\centering
	\makebox[\width]{Fonte: Autor} 
\end{figure}

As \crefrange{fig:resultadosmpc-teoss}{fig:resultadosmpc-exparx} apresentam os gráficos de resposta em malha fechada
para cada um dos experimentos aplicando os modelos obtidos.

Para fins comparativos, a \cref{fig:resultadospid} apresenta o gráfico de resposta do controlador
\acrshort{pid} apresentado na \cref{sec:controlador_pid} e a \cref{tab:resultados_mpc_e_pid}
apresenta os valores de \textit{fit} (\cref{eq:nrmse}) dos controladores \acrshort{mpc} e \acrshort{pid} 
comparando-os com os sinais de referência (\textit{set-points}).

\begin{figure}[!h]
	\caption{Resposta do controlador MPC criado a partir do modelo teórico}
	\begin{center}
		\includegraphics[width=1.00\textwidth]{./5_images/ResultadosMPC-TeoSS.eps} 
		\label{fig:resultadosmpc-teoss}
	\end{center}
	\centering
	\makebox[\width]{Fonte: Autor} 
\end{figure}

\begin{figure}[!h]
	\caption{Resposta do controlador MPC criado a partir do modelo experimental (função de transferência)}
	\begin{center}
		\includegraphics[width=1.00\textwidth]{./5_images/ResultadosMPC-ExpTF.eps} 
		\label{fig:resultadosmpc-exptf}
	\end{center}
	\centering
	\makebox[\width]{Fonte: Autor} 
\end{figure}

\begin{figure}[!h]
	\caption{Resposta do controlador MPC criado a partir do modelo experimental (ARMAX)}
	\begin{center}
		\includegraphics[width=1.00\textwidth]{./5_images/ResultadosMPC-ExpARMAX.eps} 
		\label{fig:resultadosmpc-exparmax}
	\end{center}
	\centering
	\makebox[\width]{Fonte: Autor} 
\end{figure}

\begin{figure}[!h]
	\caption{Resposta do controlador MPC criado a partir do modelo experimental (Output-Error)}
	\begin{center}
		\includegraphics[width=1.00\textwidth]{./5_images/ResultadosMPC-ExpOE.eps} 
		\label{fig:resultadosmpc-expoe}
	\end{center}
	\centering
	\makebox[\width]{Fonte: Autor} 
\end{figure}

\begin{figure}[!h]
	\caption{Resposta do controlador MPC criado a partir do modelo experimental (espeço de estados)}
	\begin{center}
		\includegraphics[width=1.00\textwidth]{./5_images/ResultadosMPC-ExpSS.eps} 
		\label{fig:resultadosmpc-expss}
	\end{center}
	\centering
	\makebox[\width]{Fonte: Autor} 
\end{figure}

\begin{figure}[!h]
	\caption{Resposta do controlador MPC criado a partir do modelo experimental (ARX)}
	\begin{center}
		\includegraphics[width=1.00\textwidth]{./5_images/ResultadosMPC-ExpARX.eps} 
		\label{fig:resultadosmpc-exparx}
	\end{center}
	\centering
	\makebox[\width]{Fonte: Autor} 
\end{figure}

\begin{figure}[!h]
	\caption{Resposta do controlador PID auto-sintonizado}
	\begin{center}
		\includegraphics[width=1.00\textwidth]{./5_images/ResultadosPID.eps} 
		\label{fig:resultadospid}
	\end{center}
	\centering
	\makebox[\width]{Fonte: Autor} 
\end{figure}

\begin{table}[!h]
	\centering
	\caption{Qualidade dos controladores \acrshort{mpc} e \acrshort{pid}}
	\label{tab:resultados_mpc_e_pid}
	\begin{tabular}{l|cc|c} \toprule
		{Modelo utilizado no controlador}              			            &	{\textit{Fit} Sensor 1}	    &	{\textit{Fit} Sensor 2}     & {\textit{Fit} médio}			    \\ \midrule
		MPC Teórico \acrshort{ss} (\cref{eq:tclab_modelo_teorico})	        &   $45.84\%$                   &   $44.07\%$                   &   $44.95\%$                       \\ 
		MPC Experimental \acrshort{oe} (\cref{tab:tclabsp-model-oe})	    &   $44.35\%$                   &   $40.57\%$                   &   $42.46\%$                       \\ 
		MPC Experimental \acrshort{armax} (\cref{tab:tclabsp-model-armax})	&   $39.92\%$                   &   $42.50\%$                   &   $41.21\%$                       \\ 
		MPC Experimental \acrshort{tf} (\cref{tab:tclabsp-model-tf})		&   $40.25\%$                   &   $34.55\%$                   &   $37.40\%$                       \\ 
		MPC Experimental \acrshort{ss} (\cref{eq:tclabsp-model-ss})			&   $38.55\%$                   &   $33.57\%$                   &   $36.06\%$                       \\ 
		MPC Experimental \acrshort{arx}	(\cref{tab:tclabsp-model-arx})		&   $43.71\%$                   &   $27.33\%$                   &   $35.52\%$                       \\ 
		PID (\cref{tab:pid_values})	                                        &   $40.13\%$                   &   $26.44\%$                   &   $33.28\%$                       \\ \bottomrule 
	\end{tabular}
	\caption*{Fonte: Autor}
\end{table}

\chapter{Resultados e conclusões}
\label{ch:resultados}

A partir da \cref{tab:resultados_mpc_e_pid} é possível observar que dentre todos os controladores analisados,
o controlador \acrshort{mpc} criado a partir do modelo em espaço de estados e obtido por abordagem teórica
foi o aquele que mostrou melhor controle da planta estudada. Mesmo quando comparado com o controlador 
\acrshort{pid} de referência, o controle \acrshort{mpc} em questão mostrou uma maior velocidade para a estabilização
dos seus múltiplos sinais de saída. Contudo, vale ressaltar que este controlador \acrshort{pid} 
foi apenas sintonizado utilizando a função de \textit{auto-tunning} disponível no \acrshort{matlab}.
Um controlador \acrshort{pid} melhor parametrizado poderia ter uma performance muito mais próxima das 
performances obtidas nos controladores \acrshort{mpc}, pois segundo \citeonline{Shaaban2013} e
\citeonline{Taysom2017} a diferença na performance destes dois controles é muito pequena quando
a planta possui pouco ou nenhum atraso (como a planta utilizada neste estudo), porém quando os atrasos
na planta aumentam, a resposta do controlador \acrshort{pid} se degrada rapidamente.

Ainda comparando os controladores \acrshort{mpc} e \acrshort{pid}, além das performances de ambos poderem ser
diferenciadas devido ao atraso do sistema, \citeonline{Taysom2017} também aponta em seu estudo que os
controladores \acrshort{mpc} geralmente apresentam uma melhor resposta para sistemas \acrshort{mimo},
além de serem mais efetivos em sistemas onde os distúrbios podem ser modelados, uma vez que o
\acrshort{mpc} consegue utilizar o modelo do processo e o modelo do distúrbio conjuntamente para um 
controle mais assertivo. Em contrapartida os controladores \acrshort{pid} são relativamente mais
fáceis de sintonizar e muito mais fáceis de se implementar, além de possuírem resposta similar ao \acrshort{mpc}
em uma grande variedade de aplicações \cite{Taysom2017}.

Outro ponto de destaque para os controladores \acrshort{mpc} é a possibilidade da configuração de
restrições de atuação, algo que é impossível de se fazer utilizando um controle \acrshort{pid}.
No caso do uso de restrições, o controlador \acrshort{mpc} impede que a(s) saída(s) da planta controlada
ultrapassem limites indesejados, fazendo com que o \acrshort{mpc} seja uma opção de controle muito
interessante quando se deseja maximizar ou minimizar valores de produção sem que algumas variáveis
do processo atinjam ou ultrapassem limites indesejados.

Observa-se através das \crefrange{fig:resultadosmpc-exptf}{fig:resultadosmpc-exparx} que mesmo utilizando
uma metodologia consolidada academicamente para a criação do controle \acrshort{mpc} através de modelos
experimentais, a resposta em malha fechada pode não ser satisfatória. Nestas figuras é possível 
observar uma oscilação indesejada nas saídas da planta, além de erro estacionário em alguns casos.
Todavia, ferramentas como o \textit{MPC design} do \acrshort{matlab} podem auxiliar no desenvolvimento
de um controle \acrshort{mpc} através de um modelo experimental que ao ser empiricamente sintonizado
pode apresentar uma resposta satisfatória em malha fechada. A \cref{fig:resultadosmpcmanual-exptf}
apresenta o gráfico em malha fechada de um controlador \acrshort{mpc} utilizando o mesmo modelo experimental em
função de transferência apresentado no capítulo anterior, porém desta vez sintonizado empiricamente utilizando
a ferramenta \textit{MPC Design}. A \cref{tab:resultados_mpc_e_pid_e_manual} inclui este novo modelo 
na tabela já apresentada anteriormente e revela que, apesar de uma sintonia empírica, este controlador
apresenta uma aproximação do \textit{set-point} bem superior em comparação aos controladores baseados em 
modelos experimentais que já havíam sido apresentados neste trabalho.

\begin{figure}[!h]
	\caption{Resposta do controlador MPC criado empiricamente a partir do modelo experimental (função de transferência)}
	\begin{center}
		\includegraphics[width=1.00\textwidth]{./5_images/ResultadosMPCManual-ExpTF.eps} 
		\label{fig:resultadosmpcmanual-exptf}
	\end{center}
	\centering
	\makebox[\width]{Fonte: Autor} 
\end{figure}

\begin{table}[!h]
	\centering
	\caption{Qualidade do controlador \acrshort{mpc} sintonizado empiricamente}
	\label{tab:resultados_mpc_e_pid_e_manual}
	\begin{tabular}{l|cc|c} \toprule
		{Modelo utilizado no controlador}              			            &	{\textit{Fit} Sensor 1}	    &	{\textit{Fit} Sensor 2}     & {\textit{Fit} médio}			    \\ \midrule
		MPC Teórico \acrshort{ss} (\cref{eq:tclab_modelo_teorico})	        &   $45.84\%$                   &   $44.07\%$                   &   $44.95\%$                       \\ 
		\textbf{MPC Experimental \acrshort{tf} (empírico)}                  &   $\mathbf{46.46\%}$          &   $\mathbf{40.68\%}$          &   $\mathbf{43.57\%}$              \\ 
		MPC Experimental \acrshort{oe} (\cref{tab:tclabsp-model-oe})	    &   $44.35\%$                   &   $40.57\%$                   &   $42.46\%$                       \\ 
		MPC Experimental \acrshort{armax} (\cref{tab:tclabsp-model-armax})	&   $39.92\%$                   &   $42.50\%$                   &   $41.21\%$                       \\ 
		MPC Experimental \acrshort{tf} (\cref{tab:tclabsp-model-tf})		&   $40.25\%$                   &   $34.55\%$                   &   $37.40\%$                       \\ 
		MPC Experimental \acrshort{ss} (\cref{eq:tclabsp-model-ss})			&   $38.55\%$                   &   $33.57\%$                   &   $36.06\%$                       \\ 
		MPC Experimental \acrshort{arx}	(\cref{tab:tclabsp-model-arx})		&   $43.71\%$                   &   $27.33\%$                   &   $35.52\%$                       \\ 
		PID (\cref{tab:pid_values})	                                        &   $40.13\%$                   &   $26.44\%$                   &   $33.28\%$                       \\ \bottomrule 
	\end{tabular}
	\caption*{Fonte: Autor}
\end{table}

Nota-se também neste controlador\footnote{
    Parâmetros encontrados empiricamente e ajustados para este controlador: Tempo de amostragem ($t_s$) = 2s; Horizonte de predição = 120;
    Horizonte de Controle = 16; $\Delta$ = 0; R = 0,4393; Q = 0,2276.
} apresentado na \cref{fig:resultadosmpcmanual-exptf} uma maior estabilidade
das variáveis manipuladas, apesar de um \textit{overshoot} maior que os demais controles.

De forma geral é possível afirmar que os objetivos deste trabalho foram alcançados com êxito, uma vez que foi
possível desenvolver e implementar um controle \acrshort{mpc} na planta desejada, comparar este controle 
com um controle \acrshort{pid} convencional e ainda documentar todas as etapas de forma que essas possa ser reproduzidas
por outros estudantes que estejam pesquisando sobre o \acrlong{mpc}.