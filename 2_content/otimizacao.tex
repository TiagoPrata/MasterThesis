\newacronym{sujeitoa}{s.a.}{sujeito a}
\chapter{Otimização}
\label{otimizacao}

\section{O problema da otimização}

Segundo \citeonline{Haugen2018} a otimização consiste em encontrar a melhor solução, e normalmente problemas de otimização são apresentados como problemas de minimização, como: "Encontre o valor ótimo de $x$ que minimize a \textit{função objetivo} $f(x)$, levando em consideração qualquer restrição sobre $x$ ou em função de $x$. A solução ótima é indicada por \simbolo{xopt}{$ x_{opt} $}{Valor ótimo de $x$ para minimizar $f(x)$}" \cite{Haugen2018}.

\citeonline{Haugen2018} ainda mostra que há várias formas de formular matematicamente um problema de otimização (minimização), mas que de forma geral, dado um modelo matemático $M$, é possível representá-lo como a minimização de $x$ para uma função $f(x)$, ou seja:

\begin{equation}
\min_{x} f(x)
\end{equation}

sujeto a (também denotado por "\acrshort{sujeitoa}") restrições, que podem ser na forma de:

\begin{itemize}
\item Restrições de desigualdade:
	\begin{equation}
	\label{min_restr_desigualdade}
	g(x) \leq 0
	\end{equation}
	onde $g$ pode ser uma função linear ou não-linear.
	
\item Restrições de igualdade:
	\begin{equation}
	\label{min_restr_igualdade}
	h(x) = 0
	\end{equation}
	onde $h$ pode ser uma função linear ou não-linear de $x$.
	
\item Limites superiores e inferiores
	\begin{equation}
	\label{min_limites}
	x_{li} \leq x \leq x_{ls}
	\end{equation}
	Onde $li$ e $ls$ indicam `limite inferior' e `limite superior', respectivamente.
\end{itemize}

Sendo que as equações \ref{min_restr_desigualdade} e \ref{min_restr_igualdade} definem restrições na relação entre as variáveis de otimização, enquanto \ref{min_limites} define as regiões limites destas mesmas variáveis.

Existem diversas métodos para encontrar a solução ótima para um problema de otimização e a sessão a seguir irá mostrar exemplos e métodos numéricos simples que nos permitirão entender em maiores detalhes como um problema de minimização pode ser resolvido. No \autoref{mpc} faremos uso das minimizações para compreender como o \acrshort{mpc} calcula valores ótimos dadas determinadas restrições em um dado horizonte de controle, pois uma maior compreensão sobre problemas de minimização pode fazer grande diferença no entendimento do controle \acrshort{mpc} em si.

\section{Algoritmos de otimização}

\subsection{Método de pesquisa em grade}

\lstinputlisting[	
	caption={[Pesquisa em grade com número escalar]
			 Pesquisa em grade com número escalar \\
		     Fonte: Autor},
	label={grid_search_scalar},
	language=Python,
	style=Python_lang]
	{./4_Algorithms/grid_search_scalar.py}
	

\subsection{Método de busca de descidas mais íngrimes}

steepest descent method

\subsection{Método de Newton}

método de newton

\subsection{Otimizadores de programação não-linear (NLP)}

NLP

\section{Algumas aplicações de otimização}

\subsection{Estimação de parâmetros}

estimação de param

\subsection{Moving Horizon Estimation}

MHE

\subsection{Model Predictive Control}

MPC