\newacronym{mimo}{MIMO}{\textit{Multiple Input Multiple Output}}
\newacronym{pid}{PID}{Proporcional Integral Derivativo}
\newacronym{mpc}{MPC}{\textit{Model Predictive Control}}
\chapter{Introdução}

%Controle de processos é fundamental na industria.	[ok]
%Objetivo do controle.								[ok]
%Otimizar além de controlar.						[ok]
%Estratégias utilizando modelos matemáticos.		[ok]
%Porque não usar PID em alguns casos?				[ok - mais ou menos]
%Conceito do MPC.									[ok]
O controle de processos tem fundamental importância no desenvolvimento industrial,
sendo amplamente utilizado em praticamente todos os segmentos da indústria,
contribuindo de maneira significativa para a maior velocidade na estabilização de sinais,
aumento da qualidade de produtos, diminuição de riscos e redução de custos operacionais.
Seu objetivo, de forma simplificada, consiste em avaliar e corrigir desvios entre um
valor desejado e o real valor medido na saída da planta para uma dada variável do
processo (ou variáveis, como em casos de processos com multiplas entradas e multiplas
saídas, também conhecidos por sua sigla em inglês \acrshort{mimo} (\textit{\acrlong{mimo}})).
A aplicação correta de estratégias de controle acarreta numa operação eficiênte da
planta, mantendo suas variáveis relevantes em condições próximas as desejadas.
A sintonia bem feita do controle auxilia também na otimização do processo,
possibilitando que o sistema opere com menor variabilidade, maximizando a produção
e minimizando a utilização de recursos. No \cref{ch:otimizacao} abordaremos mais
sobre otimização.
Modelos matemáticos podem auxiliar a estratégia de controle uma vez que um modelo da
planta ou processo pode ser utilizado para estabelecer a relação existente entre as
variáveis manipuladas e variáveis controladas, assim podendo auxiliar na predição do
comportamento dinâmico do sistema analisado. A modelagem matemática pode ser feita
utilizando dados empíricos ou através da aplicação de relações físico-químicas.

A estratégia de controle predominante na indústria é o controle \acrshort{pid}
(\acrlong{pid}) que, além de levar em consideração o efeito proporcional (P) do erro
medido, também atua em desvios relativos aos efeitos integrais (I) e derivativos (D).
Seu elevado número de aplicações deve-se a uma grande variedade de vantagens como:
sua rápida de implementação, facilidade de compreensão, disponibilidade em praticamente
todas as plataformas industriais de controle e principalmente pelo fato de não requerer
um modelo matemático do processo; porém apesar de poder ser aplicado com eficiência em
uma grande variedade de processos, o controle \acrshort{pid} aparece com menos frequência
em sistemas não-lineares, como em plantas de controle de pH, por exemplo. Em casos como
esse, outra técnica bastante utilizada na indústria (porém em proporções bem menores
que o \acrshort{pid}) pode ser utilizada: o controle \acrshort{mpc} (controle preditivo
baseado em modelo, do inglês \acrlong{mpc}). Essa técnica consiste em predizer o
comportamento futuro de um sistema utilizando para isso um modelo do mesmo. Mais
detalhes sobre o \acrshort{mpc} serão apresentados no \cref{ch:mpc}.

\section{Objetivos}

Este trabalho propõe, como objetivo principal, desenvolver um controlador \acrshort{mpc}
aplicado à um sistema didático com restrições de atuação.

Além disso os seguintes objetivos específicos também serão realizados:
\begin{itemize}
    \item Estudo do funcionamento matemático do controle \acrshort{mpc}
    \item Avaliação o desempenho do controlador \acrshort{mpc} desenvolvido em
        comparação com um controlador \acrshort{pid}
    \item Comparação entre a implementação em duas ou mais plataformas,
        como MATLAB\textsuperscript{\tiny\textregistered}\footnote{
            MATLAB\textsuperscript{\tiny\textregistered} é uma plataforma                           % Footnote
            de programação projetada especificamente para engenheiros e cientistas                  % Footnote
            onde é possível desenvolver algorítmos, realizar a análise de dados,                    % Footnote
            criar modelos, aplicações, dentre outras coisas.},                                      % Footnote
        Python\footnote{
            Python é uma linguagem de programação de alto nível,                                    % Footnote
            interpretada e orientada à objeto, muito utilizada atualmente para                      % Footnote
            aplicações nas áreas de ciência de dados, aprendizagem de máquina,                      % Footnote
            identificação de sistemas, etc.},                                                       % Footnote
        ou similares.
\end{itemize}

\section{Motivação e justificativas}

Segundo \citeonline{Parkinson2018} o processo de determinar o melhor \textit{design}
para uma aplicação ou processo é chamado de otimização, e normalmente engenheiros
costumam tentar implementar tais técnicas em seus processos visando aumentar a
eficiência diminuindo os gastos, por exemplo projetando o menor trocador de calor
que realize a transferência de calor desejada, uma ponte de menor custo para o local,
ou mesmo maximizar o rendimento de um processo químico, porém, assim como nesses
exemplos, as variáveis e limitates do processo podem ser inúmeras, fazendo com que a
tarefa de otimização se torne difícil caso o engenheiro utilize apenas uma combinação
de experiência, conhecimento, opiniões, etc. Para esses casos, ferramentas
computacionais de otimização são essenciais.

Dá-se o nome de Otimização Dinâmica ao processo de otimização que é realizado
dinamicamente ao longo do processo e, segundo \citeonline{Borrelli2017}, esta se
tornou uma ferramenta padrão na tomada de decisões numa grande variedade de áreas.
O controle \acrshort{mpc} é um modo de implementação da otimização dinâmica e a execução
deste trabalho em torno dessa técnica se deve ao fato de que ao longo dos últimos 25 anos
ela evoluiu para dominar a indústria de processos, onde tem sido utilizada em milhares
de problemas \cite{Borrelli2017}, e também ao seu crescimento em outras indústrias,
como por exemplo a descrita por \citeonline{Yakub2013} em seu estudo comparativo
mostrando a utilização do \acrshort{mpc} no controle do sistema dinâmico de um automóvel.

\section{Organização do trabalho}

% TODO: Fazer organização do trabalho