\newacronym{prbs}{PRBS}{\textit{Pseudo-Random Binary Signal}}
\newacronym{arma}{ARMA}{\textit{Auto-Regressive Moving Average}}
\newacronym{gbn}{GBN}{\textit{Generalized Binary Noise}}
\newacronym{ar}{AR}{modelo autorregressivo}
\newacronym{arx}{ARX}{modelo autorregressivo com entrada exógena}
\newacronym{armax}{ARMAX}{modelo autorregressivo com média móvel e entrada exógena}

\chapter{Metodologia}
\label{ch:metodologia}

% =====================================================================================================
% ============================================= Section ===============================================
% =====================================================================================================
\section{Modelagem experimental}
\label{sec:modelagem_experimental}

Além dos modelos teóricos descritos no capítulo anterior, neste projeto também será executada a identificação
experimental do sistema.

Segundo \citeonline{Gevers2006} a teoria de identificação de sistemas data da década de 60 e as duas
principais técnicas utilizadas na identificação de sistemas atualmente (método de identificação por
subespaço de estados e método do erro de predição) tiveram suas bases estruturadas com os trabalhos
de \apudonline{Ho1966}{Gevers2006} e de \apudonline{Astrom1965}{Gevers2006}.

De acordo com \apudonline{Dunia2008}{Pracek2012} a identificação do sistema permite que simulações
sejam desenvolvidas de modo a garantir um melhor desempenho dos sistemas de controle projetados.

As etapas para a identificação de um sistema, segundo \citeonline{Aguirre2015} são:
\begin{itemize}
    \item Testes dinâmicos e coleta de dados
    \item Escolha da representação matemática a ser utilizadas
    \item Determinação da estrutura do modelos
    \item Estimação de parâmetros
    \item Validação do modelo
\end{itemize}

As seções a seguir detalham cada uma dessas etapas.

% .....................................................................................................
% ............................................ Subsection .............................................
% .....................................................................................................
\subsection{Testes dinâmicos e coleta de dados}
\label{subsec:testes_dinamicos_e_coleta_de_dados}

Esta etapa abrange os procedimentos necessários para geração do conjunto de dados que serão utilizados
para a identificação do sistema. Algumas das atividades realizadas nessa etapa são: escolha das variáveis,
definição dos sinais de excitação, definição do período de amostragem e a execução em si dos testes.
\cite{Aguirre2015}

\begin{itemize}
    \item Período de amostragem: três regras práticas auxiliam na definição deste período: usar um tempo
        de amostragem de aproximadamente 1/10 da maior constante de tempo \apud{Gustavsson1975}{Ballin2008};
        10\% do tempo de acomodação de uma resposta degrau \cite{Ballin2008}; ou escolher um tempo de amostragem
        de 5 a 10 vezes maior do que a maior frequência de interesse contida nos dados. \cite{Aguirre2015}.
    
    \item Sinais de excitação: a escolha dos sinais de entrada pode ter grande impacto nos dados coletados,
        pois determinarão o ponto de operação do sistema e quais das suas características serão excitadas
        durante o experimento \cite{Aguirre2015}.
        Diversos tipos distintos de sinais podem ser utilizados. Entre os mais utilizados destacam-se:
        impulsivo; degrau; \acrshort{prbs} - Sinal Binário Pseudo-Aleatório (do inglês, \acrlong{prbs});
        \acrshort{arma} - Média Móvel Auto-Regressiva (do inglês, \acrlong{arma});
        \acrshort{gbn} - Ruido Binário Generalizado (do inglês, \acrlong{gbn}); soma de senóides e etc.
        \cite{Aguirre2015}
    
    \item Duração do experimento: a duração do experimento, segundo \citeonline{Garcia2005}, deveria ser
        a maior possível, uma vez que a variância das estimativas é proporcional ao inverso da duração do
        experimento, porém sob o ponto de vista prático e experimental, a duração do experimento deveria ser
        a menor possível para a obtenção de um modelo aceitável, pois ao longo do experimento o processo
        estará sujeito a perturbações extras que podem impactar na operação da planta, na qualidade dos
        produtos ou até na segurança do processo.
\end{itemize}

% .....................................................................................................
% ............................................ Subsection .............................................
% .....................................................................................................
\subsection{Escolha da representação matemática}
\label{subsec:escolha_da_representacao_matematica}

Existem diversas representações matemáticas distintas para modelos lineares, sendo que, segundo
\citeonline{Aguirre2015}, a mais utilizada é a função de transferência, porém além desta também
é importante salientar a relevância do modelo de espaço de estados (já mencionado nos capítulos
anteriores), do modelo \acrshort{ar} (\acrlong{ar}), do modelo \acrshort{arx} (\acrlong{arx}) e
do modelo \acrshort{armax} (\acrlong{armax}).

\begin{citacao}
    \text{[...]} quando se trata da modelagem obtida por meios fenomenológicos é comum que se adote a base de
    tempo contínuo, em virtude de a maioria das leis da física serem expressas nesse tempo. Por sua vez,
    quando se trata de identificação de sistemas por processos experimentais, trabalha-se com amostras
    de dados coletados a cada intervalo de tempo, nesses casos usualmente adota-se o tempo discreto.
    \apud{Garcia2005}{Favaro2012}
\end{citacao}

% .....................................................................................................
% ............................................ Subsection .............................................
% .....................................................................................................
\subsection{Determinação da estrutura do modelos}
\label{subsec:determinacao_da_estrutura_do_modelo}

Determinar a ordem de um modelo é um dos aspectos mais importantes na determinação de sua estrutura,
uma vez que, caso sua ordem seja muito menor do que a ordem efetiva do sistema real, o modelo não
refletirá a completamente sua complexidade estrutural. Analogamente, escolher um modelo que a ordem
seja muito maior do que a necessária, provavelmente causará uma estimação de parâmetros mal condicionada.
\cite{Aguirre2015}

% .....................................................................................................
% ............................................ Subsection .............................................
% .....................................................................................................
\subsection{Estimação de parâmetros}
\label{subsec:estimacao_de_parametros}

A estimação de parâmetros, segundo \apudonline{Eykhoff1974}{Favaro2012} é a determinação experimental
de valores de parâmetros que governam a dinâmica e/ou o comportamento não-linear, assumindo que a
estrutura do modelo seja conhecida.

Essa etapa começa com a escolha do algorítmo a ser utilizado \cite{Aguirre2015}. Dentre eles, os
mais amplamente empregados na literatura são: método da análise de frequência; método da resposta
transitória e método dos mínimos quadrados \cite{Favaro2012}.

% .....................................................................................................
% ............................................ Subsection .............................................
% .....................................................................................................
\subsection{Validação do modelo}
\label{subsec:validacao_do_modelo}

Em problemas de validação, a questão é tentar Determinar se um dado modelo é válido ou não e para isso,
deve-se simulá-lo sem qualquer ajuste adicional e compará-lo a dados medidos em testes diferentes daquele
usado no desenvolvimento da sintonia do mesmo. Ao se obter um conjunto de modelos, deve-se verificar se
eles incorporaram as características de interesse do sistema original \cite{Aguirre2015}.

\citeonline{Aguirre2015} em seu livro apresenta diversas ferramentas para auxiliar na validação dos modelos.

% =====================================================================================================
% ============================================= Section ===============================================
% =====================================================================================================
\section{Desenvolvimento do controlador}
\label{sec:desenvolvimento_do_controlador}

% TODO descrever

% =====================================================================================================
% ============================================= Section ===============================================
% =====================================================================================================
\section{Análise de resultados}
\label{sec:analise_de_resultados}

% TODO descrever