
% Pacotes básicos 
\usepackage{lmodern}			% Usa a fonte Latin Modern			
\usepackage[T1]{fontenc}		% Seleção de códigos de fonte.
\usepackage[utf8]{inputenc}		% Codificação do documento (conversão automática dos acentos)
\usepackage{indentfirst}		% Identa o primeiro parágrafo de cada seção.
\usepackage{color}				% Controle das cores
\usepackage{graphicx}			% Inclusão de gráficos
\usepackage{microtype} 			% para melhorias de justificação

% Pacote para cores em tabelas
\usepackage{colortbl}
\definecolor{destaqueVerde}{rgb}{0.92,1,0.92}
\definecolor{destaqueVermelho}{rgb}{1,0.92,0.92}

% Pacote de glossário
\usepackage{./9_Extras/styles/anbtex2-glossario}

% Pacotes de citações
% \usepackage[brazilian,hyperpageref]{backref}	 % Paginas com as citações na bibliografia
% % Usado sem a opção hyperpageref de backref
\renewcommand{\backrefpagesname}{Citado na(s) página(s):~}
% Texto padrão antes do número das páginas
\renewcommand{\backref}{}
% Define os textos da citação
\renewcommand*{\backrefalt}[4]{
	\ifcase #1 %
		Nenhuma citação no texto.%
	\or
		Citado na página #2.%
	\else
		Citado #1 vezes nas páginas #2.%
	\fi}%
\usepackage[alf,abnt-emphasize=bf]{abntex2cite}	% Citações padrão ABNT

% Pacotes para algoritmos
\usepackage{listings}
\renewcommand{\lstlistlistingname}{Lista de códigos-fonte} 
\renewcommand{\lstlistingname}{Código-fonte} 
\newlistof{lstlistoflistings}{lol}{\lstlistlistingname}

% Configuracoes gerais
%\lstset{ 
%  backgroundcolor=\color{white},   % choose the background color; you must add \usepackage{color} or \usepackage{xcolor}; should come as last argument
%  basicstyle=\footnotesize,        % the size of the fonts that are used for the code
%  breakatwhitespace=false,         % sets if automatic breaks should only happen at whitespace
%  breaklines=true,                 % sets automatic line breaking
%  captionpos=b,                    % sets the caption-position to bottom
%  commentstyle=\color{mygreen},    % comment style
%  deletekeywords={...},            % if you want to delete keywords from the given language
%  escapeinside={\%*}{*)},          % if you want to add LaTeX within your code
%  extendedchars=true,              % lets you use non-ASCII characters; for 8-bits encodings only, does not work with UTF-8
%  firstnumber=1000,                % start line enumeration with line 1000
%  frame=single,	                   % adds a frame around the code
%  keepspaces=true,                 % keeps spaces in text, useful for keeping indentation of code (possibly needs columns=flexible)
%  keywordstyle=\color{blue},       % keyword style
%  morekeywords={*,...},            % if you want to add more keywords to the set
%  numbers=left,                    % where to put the line-numbers; possible values are (none, left, right)
%  numbersep=5pt,                   % how far the line-numbers are from the code
%  numberstyle=\tiny\color{mygray}, % the style that is used for the line-numbers
%  rulecolor=\color{black},         % if not set, the frame-color may be changed on line-breaks within not-black text (e.g. comments (green here))
%  showspaces=false,                % show spaces everywhere adding particular underscores; it overrides 'showstringspaces'
%  showstringspaces=false,          % underline spaces within strings only
%  showtabs=false,                  % show tabs within strings adding particular underscores
%  stepnumber=2,                    % the step between two line-numbers. If it's 1, each line will be numbered
%  stringstyle=\color{mymauve},     % string literal style
%  tabsize=2,	                   % sets default tabsize to 2 spaces
%  title=\lstname                   % show the filename of files included with \lstinputlisting; also try caption instead of title
%}

\definecolor{listings_keyword}{RGB}{30,80,179}
\definecolor{listings_comment}{RGB}{82,151,82}
\definecolor{listings_identifier}{RGB}{0,0,0}
\definecolor{listings_string}{RGB}{225,89,89}
\definecolor{listings_emphs}{RGB}{159,77,153}
\lstset{
  belowcaptionskip=1\baselineskip,
  breaklines=true,
  frame=tbL,
  showstringspaces=false,
  basicstyle=\footnotesize\ttfamily,
  keywordstyle=\bfseries\color{listings_keyword},
  commentstyle=\itshape\color{listings_comment},
  identifierstyle=\color{listings_identifier},
  emphstyle=\color{listings_emphs},
  stringstyle=\color{listings_string},
  captionpos=b,
  escapechar=\§,
}

% Ajustes para linguagem Python
\lstdefinestyle{Python_lang}{
  language=Python,
  morekeywords={as},
  emph={import,return},
}


\usepackage{amsmath, amssymb}
% \usepackage{pgf}
\usepackage{pdfpages}

% Pacote para frações diagonais
\usepackage{xfrac}

% Pacote para aprimoramento do caption
\usepackage{caption}

% Pacote para citar nome das seções ao invés dos números
\usepackage{nameref}

% Pacote para highlight
\usepackage{soul}

% Configurando matrizes com colunas
\usepackage{etoolbox}
\let\bbordermatrix\bordermatrix
\patchcmd{\bbordermatrix}{8.75}{4.75}{}{}
\patchcmd{\bbordermatrix}{\left(}{\left[}{}{}
\patchcmd{\bbordermatrix}{\right)}{\right]}{}{}
\definecolor{matrixtitle}{gray}{0.65}

% Cria comando para referência de anexos. https://github.com/abntex/abntex2/issues/76
\newcommand{\crefanexo}[1]{\hyperref[#1]{anexo~\ref{#1}}}
\newcommand{\Crefanexo}[1]{\hyperref[#1]{Anexo~\ref{#1}}}

% CONFIGURAÇÕES DE PACOTES

% Informações de dados para CAPA e FOLHA DE ROSTO
\titulo{Controle Preditivo Baseado em Modelo (MPC) aplicado a uma planta didática}
\autor{Tiago Correa Prata}
\local{São Paulo}
\data{2020}
\orientador{Prof. Dr. Alexandre Brincalepe Campo}
%\coorientador{----}
\instituicao{Instituto Federal de São Paulo - IFSP}
\tipotrabalho{Tese (Mestrado)}
% O preambulo deve conter o tipo do trabalho, o objetivo, 
% o nome da instituição e a área de concentração 
\preambulo{Projeto de pesquisa apresentado ao Instituto Federal de Educação, Ciência
e Tecnologia de São Paulo para a qualificação no programa de mestrado em engenharia
de automação e controle.}


\makeatletter
\hypersetup{
     	%pagebackref=true,
		pdftitle={\@title}, 
		pdfauthor={\@author},
    	pdfsubject={\imprimirpreambulo},
	    pdfcreator={LaTeX},
		pdfkeywords={MPC}{Model Predictive Control}{Controle Avançado}, 
		colorlinks=true,       		% false: boxed links; true: colored links
    	linkcolor=PDF_blue,          	% color of internal links
    	citecolor=PDF_blue,        		% color of links to bibliography
    	filecolor=magenta,      		% color of file links
		urlcolor=PDF_blue,
		bookmarksdepth=4
}
\makeatother

\usepackage[nameinlink, brazilian]{cleveref}
