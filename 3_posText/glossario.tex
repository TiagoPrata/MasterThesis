% ----------------------------------------------------------
% Glossário
% ----------------------------------------------------------

% ---
% Define nome e preâmbulo do glossário
% ---
\phantompart
%\renewcommand{\glossaryname}{Glossário}  A opção babel do glossaries faz a tradução.
\renewcommand{\glossarypreamble}{Esta é a descrição do glossário. Experimente
visualizar outros estilos de glossários, como o \texttt{altlisthypergroup},
por exemplo.\\
\\}

% ---
% Traduções para o ambiente glossaries
% ---  
% A opção babel do glossaries faz a tradução.

%\providetranslation{Glossary}{Glossário}
%\providetranslation{Acronyms}{Siglas}
%\providetranslation{Notation (glossaries)}{Notação}
%\providetranslation{Description (glossaries)}{Descrição}
%\providetranslation{Symbol (glossaries)}{Símbolo}
%\providetranslation{Page List (glossaries)}{Lista de Páginas}
%\providetranslation{Symbols (glossaries)}{Símbolos}
%\providetranslation{Numbers (glossaries)}{Números} 
% ---

% ---
% Estilo de glossário
% ---
% \setglossarystyle{index}
% \setglossarystyle{altlisthypergroup}
 \setglossarystyle{tree}  % Já selecionado no arquivo .sty 


% ---
% Imprime o glossário
% ---
\cleardoublepage
\phantomsection
\addcontentsline{toc}{chapter}{\glossaryname}
\printglossaries % imprime todas as entradas

\imprimirglossario  % não imprime acronimos (siglas) e nem simbolos
% ---