\begin{apendicesenv}

% Imprime uma página indicando o início dos apêndices
\partapendices

% =====================================================================================================
% ============================================= Chapter ===============================================
% =====================================================================================================
\chapter{Códigos-fonte}
\label{ch:codigos_extras}

\section{Método de pesquisa em grade}

\lstinputlisting[	
	caption={Pesquisa em grade com número escalar},
	captionpos=t,
	label={lst:grid_search_scalar},
	language=Python,
	style=Python_lang]
	{./4_Codes/grid_search_scalar.py}
	\begin{center}
		\makebox[\width]{Fonte: Autor, adaptado de \citeonline{Haugen2018}}
	\end{center}

\section{Método de pesquisa em grade com duas variáveis}

\lstinputlisting[	
	caption={Pesquisa em grade com duas variáveis},
	captionpos=t,
	label={lst:grid_search_vectorial},
	language=Python,
	style=Python_lang]
	{./4_Codes/grid_search_vectorial.py}
	\begin{center}
		\makebox[\width]{Fonte: Autor, adaptado de \citeonline{Haugen2018}}
	\end{center}

\section{Exemplo do método \acrlong{mhe}}

% (Brincalepe): O código da página 44 deve ser deslocado para o apêndice.
% (cont.) Se quiser, destaque apenas alguma parte mais relevante e mantenha no corpo do texto.
\lstinputlisting[	
	caption={Exemplo do método \acrlong{mhe}},
	captionpos=t,
	label={lst:mhe_example},
	language=Python,
	style=Python_lang]
	{./4_Codes/mhe_example.py}
	\begin{center}
		\makebox[\width]{Fonte: Autor, adaptado de \citeonline{Haugen2018}}
	\end{center}

\section{Exemplo de aplicação do \acrlong{mpc}}

\lstinputlisting[	
	caption={Exemplo de aplicação do \acrlong{mpc}},
	captionpos=t,
	label={lst:mpc_example},
	language=Python,
	style=Python_lang]
	{./4_Codes/mpc_example.py}
	\begin{center}
		\makebox[\width]{Fonte: Autor, adaptado de \citeonline{Haugen2018}}
	\end{center}
% (Brincalepe): O código da página 58 também pode ser deslocado para o Apêndice.
% (cont.) Seria interessante descrevê-lo no corpo de texto através de um diagrama.

\section{Criação de diversos modelos experimentais do \acrshort{tclabsp}}
\label{sec:tclabsp-models-creation}

\lstinputlisting[	
	caption={Criação de diversos modelos experimentais do \acrshort{tclabsp}},
	captionpos=t,
    label={lst:tclabsp-models-creation},
    firstline=56,
    lastline=387,
    language=Matlab,
	style=Matlab_lang]
	{./4_codes/modelo_experimental/RunMe.m}
	\begin{center}
		\makebox[\width]{Fonte: Autor}
    \end{center}

\section{Identificação do tempo de subida do TCLabSP}
\label{sec:tclabsp_tempo_de_subida}

\lstinputlisting[	
	caption={Calcula tempo de subida de primeira ordem do TCLabSP},
	captionpos=t,
	label={lst:rise_time},
	language=Matlab,
	style=Matlab_lang]
	{./4_Codes/rise_time/calc_rise_time.m}
	\begin{center}
		\makebox[\width]{Fonte: Autor}
	\end{center}

\lstinputlisting[	
	caption={Função auxiliar: Combina experimentos do TCLabSP},
	captionpos=t,
	label={lst:combina_exp},
	language=Matlab,
	style=Matlab_lang]
	{./4_Codes/rise_time/combinar_experimentos.m}
	\begin{center}
		\makebox[\width]{Fonte: Autor}
	\end{center}

\lstinputlisting[	
	caption={Função auxiliar: Calcula média de uma região},
	captionpos=t,
	label={lst:media_regiao},
	language=Matlab,
	style=Matlab_lang]
	{./4_Codes/rise_time/media_regiao.m}
	\begin{center}
		\makebox[\width]{Fonte: Autor}
	\end{center}

\lstinputlisting[	
	caption={Plota tempo de subida},
	captionpos=t,
	label={lst:plot_rise_time},
	language=Matlab,
	style=Matlab_lang]
	{./4_Codes/rise_time/plot_rise_time.m}
	\begin{center}
		\makebox[\width]{Fonte: Autor}
	\end{center}

\section{Encontrar Delta amostral utilizando autocovariância}
\label{sec:tclabsp_find_delta}

\lstinputlisting[	
	caption={Encontra o Delta e o Ts},
	captionpos=t,
	label={lst:calc_delta},
	language=Matlab,
	style=Matlab_lang]
	{./4_Codes/find_delta/calc_delta_and_Ts.m}
	\begin{center}
		\makebox[\width]{Fonte: Autor}
	\end{center}

\lstinputlisting[	
	caption={Função auxiliar: Encontra Delta},
	captionpos=t,
	label={lst:find_delta},
	language=Matlab,
	style=Matlab_lang]
	{./4_Codes/find_delta/find_delta.m}
	\begin{center}
		\makebox[\width]{Fonte: Autor}
	\end{center}

\lstinputlisting[	
	caption={Função auxiliar: Calcula autocovariância},
	captionpos=t,
	label={lst:autocorr_min},
	language=Matlab,
	style=Matlab_lang]
	{./4_Codes/find_delta/autocorr_min.m}
	\begin{center}
		\makebox[\width]{Fonte: Autor}
	\end{center}

\lstinputlisting[	
	caption={Plota autocovariância},
	captionpos=t,
	label={lst:plot_autocorr},
	language=Matlab,
	style=Matlab_lang]
	{./4_Codes/find_delta/plot_autocorr.m}
	\begin{center}
		\makebox[\width]{Fonte: Autor}
	\end{center}

% =====================================================================================================
% ============================================= Chapter ===============================================
% =====================================================================================================
\chapter{Escolha da frequência de amostragem utilizando autocovariância}
\label{ch:sampling_time_using_autocorrelation}

A regra prática para a escolha da frequência de amostragem de sinal descrita por \citeonline{Aguirre2015}
e citada na \cref{subsec:periodo_de_amostragem} pode nem sempre ajudar muito, uma vez que é possível
que não se tenha nenhum conhecimento prévio do sinal para poder aplicar a regra de escolher uma frequência
de 5 a 10 vezes maior que a frequência de interesse \cite{Aguirre2015}. Para casos assim, \citeonline{Aguirre2015}
descreve uma outra técnica, que será apresentada nos parágrafos a seguir. 

Assume-se que o sinal já amostrado, $y^*(k)$, tenha sido obtido utilizando-se um tempo
de amostragem muito pequeno, ou seja, o sinal está superamostrado. Sendo assim, deseja-se encontrar o
valor $\Delta$ que descreva a fração pela qual o $y^*(k)$ pode ser decimado a fim de obter um sinal de trabalho
$y(k) = y^*(\Delta k)$, ou seja, o sinal decimado que ainda mantem as características originais do sinal $y^*(k)$.

Para fazer isso é necessário verificar o nível de correlação entre observações adjacentes do sinal
$y^*(k)$. Quanto mais superamostrado estiver o sinal $y^*(k)$, maior será a redundância entre duas
observações consecutivas. Este nível de correlação pode ser obtido calculando as funções de
autocovariância linear e não-linear (mostradas na \cref{eq:autocorrelation}) do sinal original.

\begin{subequations}
    \label{eq:autocorrelation}
    \begin{align}
		r_{y^*}(\tau) &= \mathrm{E} \left[(y^*(k) - \overline{y^*(k)}) (y^*(k - \tau) - \overline{y^*(k)})\right]		\\
		r_{y^{*2'}}(\tau) &= \mathrm{E} \left[(y^{*2}(k) - \overline{y^{*2}(k)}) (y^{*2}(k - \tau) - \overline{y^{*2}(k)})\right]
    \end{align}
\end{subequations}

$\mathrm{E}[\cdot]$ indica a esperança matemática, porém, sendo $y^*(k)$ ergótico, $\mathrm{E}[\cdot]$
pode ser substituída pela média temporal.

Após calculados $r_{y^*}(\tau)$ e $r_{y^{*2'}}(\tau)$, determinam-se seus primeiros mínimos,
$\tau_{y^*}$ e $\tau_{y^{*2'}}$, respectivamente. O menor desses mínimos passa a ser o valor de trabalho
$\tau_{m}^{*}$, ou seja, $\tau_{m}^{*} = \mathrm{min} \left[ \tau_{y^*} , \tau_{y^{*2'}} \right]$.

Por fim, deseja-se escolha $\Delta$ de forma que as funções de autocovariância do sinal decimado $y(k) = y^*(k)$
satisfaçam

\begin{equation}
    \label{eq:tau_m}
    10 \leq \tau_m \leq 20
\end{equation}

\noindent
sendo que $\tau_m$ é definido pelo sinal decimado $y(k)$ de maneira análoga a $\tau_m^*$ para o
sinal $y^*(k)$. Os limites inferior e superior da \cref{eq:tau_m} podem ser relaxados para
$5$ e $25$, respectivamente.

% =====================================================================================================
% ============================================= Section ===============================================
% =====================================================================================================
\section{Aplicação prática}
\label{ch:using_sampling_time_with_autocorrelation}

As \cref{fig:rise_time_sensor1,fig:rise_time_sensor2} da \cref{subsec:periodo_de_amostragem}
apresentam a resposta dos sensores de temperatura do \acrshort{tclabsp} quando o Aquecedor 1 é
excitado com um degrau de 0 à 50\%.

Após alguns testes preliminares, foi observado que o sistema apresentava uma resposta lenta
e que, portanto, uma coleta de dados com tempo de amostragem de $0.25s$ seria suficiente para
superamostrar os dados e garantir que todas as características de resposta do sistema tenham
sido amostradas.

Calcula-se então as funções de autocovariância apresentadas na \cref{eq:autocorrelation} para os
sensores de temperatura 1 e 2, como apresentado na \cref{fig:autocorrelationS1S2}.

\begin{figure}[h]
	\caption{Funções de autocovariância dos Sensores de Temperatura 1 e 2}
	\begin{center}
		\includegraphics[width=0.95\textwidth]{./5_images/AutocorrelationS1S2.png} 
		\label{fig:autocorrelationS1S2}
	\end{center}
	\centering
	\makebox[\width]{Fonte: \citeonline{Prata2019}} 
\end{figure}

Para cada um dos sensores, encontramos então o valor mínimo das funções de autocovariância
($\tau_{y^{*}}$ e $\tau_{y^{*2'}}$), e a partir delas determinamos $\tau_m^*$
($\tau_{m}^{*} = \mathrm{min} \left[ \tau_{y^*} , \tau_{y^{*2'}} \right]$).
O resultado é apresentado na \cref{tab:tau_s1s2}.

\begin{table}[h]
	\centering
	\caption{Mínimos das funções de autocovariância}
	\label{tab:tau_s1s2}
	\begin{tabular}{llll} \toprule
		{Sensor}		& {$\tau_{y^{*}}$}		& {$\tau_{y^{*2'}}$}		& {$\pmb{\tau}_{\pmb{m}}^{\pmb{*}}$}		\\ \midrule
		Sensor 1		& $7015$				& $6925$					& $\pmb{6}\pmb{9}\pmb{2}\pmb{5}$			\\
		Sensor 2		& $6735$				& $6621$					& $\pmb{6}\pmb{6}\pmb{2}\pmb{1}$			\\ \bottomrule
	\end{tabular}
	\caption*{Fonte: Autor}
\end{table}

Conhecendo $\tau_{m}^{*}$ é possível encontrar um valor de $\Delta$ para obter o sinal decimado
$y(k) = y^*(\Delta k)$, tal que o valor mínimo da suas funções de autocovariância satisfaçam
\cref{eq:tau_m}.

Uma demonstração do efeito da variação do $\Delta$ no tempo de amostragem do sinal
pode ver vista na \cref{tab:delta_action}

\begin{table}[h]
	\centering
	\caption{Efeito do $\Delta$ no tempo de amostragem}
	\label{tab:delta_action}
	\begin{tabular}{cccccc} \toprule
		{k}			& {$\Delta=1$}		& {$\Delta=2$}		& {$\cdots$}	& {$\Delta=4$}	& {$\cdots$} 	\\ \midrule
		1			& $0.00$			& $0.00$			& \hfill		& $0.00$		& \hfill		\\
		2			& $0.25$			& $0.50$			& \hfill		& $1.00$		& \hfill		\\
		3			& $0.50$			& $1.00$			& \hfill		& $2.00$		& \hfill		\\
		4			& $0.75$			& $0.50$			& \hfill		& $3.00$		& \hfill		\\
		5			& $1.00$			& $2.00$			& \hfill		& $4.00$		& \hfill		\\ 
		$\vdots$	& $\vdots$			& $\vdots$			& \hfill		& $\vdots$		& \hfill		\\ \bottomrule 
	\end{tabular}
	\caption*{Fonte: Autor}
\end{table}

Através de um algoritmo\footnote{
	\textit{Scripts} utilizados no \cref{sec:tclabsp_find_delta}.
} foi possível testar diferentes valores de $\Delta$ a fim de encontrar
aquele que primeiro satisfizesse a \cref{eq:tau_m}, e por meio deste algoritmo encontrou-se
os valores de $\Delta$ e de $T_s$ (tempo de amostragem) indicados da \cref{eq:delta_and_ts}.

\begin{subequations}
    \label{eq:delta_and_ts}
    \begin{align}
		\Delta &= 253	\\
		T_s &= 63.25s
    \end{align}
\end{subequations}

% =====================================================================================================
% ============================================= Chapter ===============================================
% =====================================================================================================
\chapter{Gráficos dos testes de validação dos modelos experimentais}
\label{ch:graficos_testes_de_validacao}

\begin{figure}
	\caption{Modelo: função de transferência - Teste de validação}
	\begin{center}
		\includegraphics[width=1.00\textwidth]{./5_images/tclabsp-models-TF-compare.png} 
		\label{fig:tclabsp-models-tf-compare}
	\end{center}
	\centering
	\makebox[\width]{Fonte: Autor} 
\end{figure}

\begin{figure}
	\caption{Modelo: função de transferência - Análise de resíduos}
	\begin{center}
		\includegraphics[width=1.00\textwidth]{./5_images/tclabsp-models-TF-resid.png} 
		\label{fig:tclabsp-models-tf-resid}
	\end{center}
	\centering
	\makebox[\width]{Fonte: Autor} 
\end{figure}

\begin{figure}[h]
	\caption{Modelos experimentais de espeço de estados}
	\begin{center}
		\includegraphics[width=1.00\textwidth]{./5_images/tclabsp-models-SS.png} 
		\label{fig:tclabsp-models-ss}
	\end{center}
	\centering
	\makebox[\width]{Fonte: Autor} 
\end{figure}

\begin{figure}
	\caption{Modelo: espaço de estados - Teste de validação}
	\begin{center}
		\includegraphics[width=1.00\textwidth]{./5_images/tclabsp-models-SS-compare.png} 
		\label{fig:tclabsp-models-ss-compare}
	\end{center}
	\centering
	\makebox[\width]{Fonte: Autor} 
\end{figure}

\begin{figure}[h]
	\caption{Modelos experimentais ARX}
	\begin{center}
		\includegraphics[width=1.00\textwidth]{./5_images/tclabsp-models-ARX.png} 
		\label{fig:tclabsp-models-arx}
	\end{center}
	\centering
	\makebox[\width]{Fonte: Autor} 
\end{figure}

\begin{figure}
	\caption{Modelo: ARX - Teste de validação}
	\begin{center}
		\includegraphics[width=1.00\textwidth]{./5_images/tclabsp-models-ARX-compare.png} 
		\label{fig:tclabsp-models-arx-compare}
	\end{center}
	\centering
	\makebox[\width]{Fonte: Autor} 
\end{figure}

\begin{figure}
	\caption{Modelo: ARX - Análise de resíduos}
	\begin{center}
		\includegraphics[width=1.00\textwidth]{./5_images/tclabsp-models-ARX-resid.png} 
		\label{fig:tclabsp-models-arx-resid}
	\end{center}
	\centering
	\makebox[\width]{Fonte: Autor} 
\end{figure}

\begin{figure}[h]
	\caption{Modelos experimentais ARMAX}
	\begin{center}
		\includegraphics[width=1.00\textwidth]{./5_images/tclabsp-models-ARMAX.png} 
		\label{fig:tclabsp-models-armax}
	\end{center}
	\centering
	\makebox[\width]{Fonte: Autor} 
\end{figure}

\begin{figure}
	\caption{Modelo: ARMAX - Teste de validação}
	\begin{center}
		\includegraphics[width=1.00\textwidth]{./5_images/tclabsp-models-ARMAX-compare.png} 
		\label{fig:tclabsp-models-armax-compare}
	\end{center}
	\centering
	\makebox[\width]{Fonte: Autor} 
\end{figure}

\begin{figure}
	\caption{Modelo: ARMAX - Análise de resíduos}
	\begin{center}
		\includegraphics[width=1.00\textwidth]{./5_images/tclabsp-models-ARMAX-resid.png} 
		\label{fig:tclabsp-models-armax-resid}
	\end{center}
	\centering
	\makebox[\width]{Fonte: Autor} 
\end{figure}

\begin{figure}[h]
	\caption{Modelos experimentais OE}
	\begin{center}
		\includegraphics[width=1.00\textwidth]{./5_images/tclabsp-models-OE.png} 
		\label{fig:tclabsp-models-oe}
	\end{center}
	\centering
	\makebox[\width]{Fonte: Autor} 
\end{figure}

\begin{figure}
	\caption{Modelo: OE - Teste de validação}
	\begin{center}
		\includegraphics[width=1.00\textwidth]{./5_images/tclabsp-models-OE-compare.png} 
		\label{fig:tclabsp-models-oe-compare}
	\end{center}
	\centering
	\makebox[\width]{Fonte: Autor} 
\end{figure}

\begin{figure}
	\caption{Modelo: OE - Análise de resíduos}
	\begin{center}
		\includegraphics[width=1.00\textwidth]{./5_images/tclabsp-models-OE-resid.png} 
		\label{fig:tclabsp-models-oe-resid}
	\end{center}
	\centering
	\makebox[\width]{Fonte: Autor} 
\end{figure}

\begin{figure}[h]
	\caption{Modelos experimentais BJ}
	\begin{center}
		\includegraphics[width=1.00\textwidth]{./5_images/tclabsp-models-BJ.png} 
		\label{fig:tclabsp-models-bj}
	\end{center}
	\centering
	\makebox[\width]{Fonte: Autor} 
\end{figure}

\begin{figure}
	\caption{Modelo: BJ - Teste de validação}
	\begin{center}
		\includegraphics[width=1.00\textwidth]{./5_images/tclabsp-models-BJ-compare.png} 
		\label{fig:tclabsp-models-bj-compare}
	\end{center}
	\centering
	\makebox[\width]{Fonte: Autor} 
\end{figure}

\begin{figure}
	\caption{Modelo: BJ - Análise de resíduos}
	\begin{center}
		\includegraphics[width=1.00\textwidth]{./5_images/tclabsp-models-BJ-resid.png} 
		\label{fig:tclabsp-models-bj-resid}
	\end{center}
	\centering
	\makebox[\width]{Fonte: Autor} 
\end{figure}

% =====================================================================================================
% ============================================= Chapter ===============================================
% =====================================================================================================
\chapter{MPC Experimental Empírico}
\label{ch:mpc_experimental_empirico}

\section{Telas do \textit{MPC Design}}

\begin{figure}[!h]
	\caption{MPC Design - Tela principal}
	\begin{center}
		\includegraphics[width=1.00\textwidth]{./5_images/mpcdesign_main.png} 
		\label{fig:mpcdesign_main}
	\end{center}
	\centering
	\makebox[\width]{Fonte: Autor} 
\end{figure}

\begin{figure}[!h]
	\caption{MPC Design - Limitantes}
	\begin{center}
		\includegraphics[width=0.60\textwidth]{./5_images/mpcdesign_constraints.png} 
		\label{fig:mpcdesign_constraints}
	\end{center}
	\centering
	\makebox[\width]{Fonte: Autor} 
\end{figure}

\begin{figure}[!h]
	\caption{MPC Design - Pesos}
	\begin{center}
		\includegraphics[width=0.60\textwidth]{./5_images/mpcdesign_weights.png} 
		\label{fig:mpcdesign_weights}
	\end{center}
	\centering
	\makebox[\width]{Fonte: Autor} 
\end{figure}

\begin{figure}[!h]
	\caption{MPC Design - Cenário}
	\begin{center}
		\includegraphics[width=0.60\textwidth]{./5_images/mpcdesign_scenarios.png} 
		\label{fig:mpcdesign_scenarios}
	\end{center}
	\centering
	\makebox[\width]{Fonte: Autor} 
\end{figure}

\clearpage

\section{Tabelas de performance}

\begin{table}[!h]
	\centering
	\caption{Performance dos controladores \acrshort{mpc} e \acrshort{pid} - IAE}
	\label{tab:resultados_mpc_e_pid_manual_iae}
	\begin{tabular}{l|cc|c|c} \toprule
		{Modelo utilizado no controlador}              			            &	\begin{tabular}[c]{@{}c@{}}Sensor 1\\(x$10^3$)\end{tabular}		&	\begin{tabular}[c]{@{}c@{}}Sensor 2\\(x$10^3$)\end{tabular}     		& \begin{tabular}[c]{@{}c@{}}Média\\(x$10^3$)\end{tabular}		& 	\begin{tabular}[c]{@{}c@{}}Média\\normalizada\end{tabular}		\\ \midrule
		MPC Teórico \acrshort{ss} (\cref{eq:tclab_modelo_teorico})	        &   $38.07$           												&   $27.94$          														&   $33.00$														&   $0.000$             											\\ 
		\textbf{MPC Exp. \acrshort{tf} (Empírico)}                  		&   $\mathbf{41.17}$          										&   $\mathbf{33.90}$          												&   $\mathbf{37.53}$											&   $\mathbf{0.133}$												\\ 
		MPC Experimental \acrshort{oe} (\cref{tab:tclabsp-model-oe})	    &   $45.53$           												&   $42.16$          														&   $43.85$														&   $0.318$             											\\ 
		MPC Experimental \acrshort{armax} (\cref{tab:tclabsp-model-armax})	&   $58.16$           												&   $37.27$          														&   $47.72$														&   $0.432$             											\\ 
		MPC Experimental \acrshort{tf} (\cref{tab:tclabsp-model-tf})		&   $56.90$           												&   $55.08$          														&   $55.99$														&   $0.674$             											\\ 
		PID (\cref{tab:pid_values})	                                        &   $53.90$           												&   $59.10$          														&   $56.50$														&   $0.690$             											\\ 
		MPC Experimental \acrshort{arx}	(\cref{tab:tclabsp-model-arx})		&   $59.13$           												&   $63.35$          														&   $61.24$														&   $0.829$             											\\ 
		MPC Experimental \acrshort{ss} (\cref{eq:tclabsp-model-ss})			&   $75.99$           												&   $58.12$          														&   $67.06$														&   $1.000$             											\\ \bottomrule 
	\end{tabular}
	\caption*{Fonte: Autor}
\end{table}   

\begin{table}[!h]
	\centering
	\caption{Performance dos controladores \acrshort{mpc} e \acrshort{pid} - ISE}
	\label{tab:resultados_mpc_e_pid_manual_ise}
	\begin{tabular}{l|cc|c|c} \toprule
		{Modelo utilizado no controlador}              			            &	\begin{tabular}[c]{@{}c@{}}Sensor 1\\(x$10^4$)\end{tabular}		&	\begin{tabular}[c]{@{}c@{}}Sensor 2\\(x$10^4$)\end{tabular}     		& \begin{tabular}[c]{@{}c@{}}Média\\(x$10^4$)\end{tabular}			& 	\begin{tabular}[c]{@{}c@{}}Média\\normalizada\end{tabular}		\\ \midrule
		MPC Teórico \acrshort{ss} (\cref{eq:tclab_modelo_teorico})	        &   $62.95$           												&   $29.03$          														&   $45.99$															&   $0.000$															\\ 
		\textbf{MPC Exp. \acrshort{tf} (Empírico)}                  		&   $\mathbf{61.43}$          										&   $\mathbf{32.56}$          												&   $\mathbf{47.00}$												&   $\mathbf{0.058}$												\\ 
		MPC Experimental \acrshort{oe} (\cref{tab:tclabsp-model-oe})	    &   $66.46$           												&   $32.78$          														&   $49.62$															&   $0.207$             											\\ 
		MPC Experimental \acrshort{armax} (\cref{tab:tclabsp-model-armax})	&   $77.46$           												&   $30.68$          														&   $54.07$															&   $0.462$             											\\ 
		MPC Experimental \acrshort{tf} (\cref{tab:tclabsp-model-tf})		&   $76.61$           												&   $39.75$          														&   $58.18$															&   $0.698$             											\\ 
		MPC Experimental \acrshort{arx}	(\cref{tab:tclabsp-model-arx})		&   $67.98$           												&   $49.01$          														&   $58.50$															&   $0.716$             											\\ 
		MPC Experimental \acrshort{ss} (\cref{eq:tclabsp-model-ss})			&   $81.04$           												&   $40.95$          														&   $61.00$															&   $0.859$             											\\ 
		PID (\cref{tab:pid_values})	                                        &   $76.83$           												&   $50.06$          														&   $63.45$															&   $1.000$             											\\ \bottomrule 
	\end{tabular}
	\caption*{Fonte: Autor}
\end{table}

\begin{table}[!h]
	\centering
	\caption{Performance dos controladores \acrshort{mpc} e \acrshort{pid} - ITAE}
	\label{tab:resultados_mpc_e_pid_manual_itae}
	\begin{tabular}{l|cc|c|c} \toprule
		{Modelo utilizado no controlador}              			            &	\begin{tabular}[c]{@{}c@{}}Sensor 1\\(x$10^6$)\end{tabular}		&	\begin{tabular}[c]{@{}c@{}}Sensor 2\\(x$10^6$)\end{tabular}     		& \begin{tabular}[c]{@{}c@{}}Média\\(x$10^6$)\end{tabular}			& 	\begin{tabular}[c]{@{}c@{}}Média\\normalizada\end{tabular}		\\ \midrule
		MPC Teórico \acrshort{ss} (\cref{eq:tclab_modelo_teorico})	        &   $60.99$           												&   $38.11$          														&   $49.55$ 														&   $0.000$            												\\ 
		\textbf{MPC Exp. \acrshort{tf} (Empírico)}                  		&   $\mathbf{63.90}$          										&   $\mathbf{46.74}$          												&   $\mathbf{55.32}$												&   $\mathbf{0.085}$												\\ 
		MPC Experimental \acrshort{oe} (\cref{tab:tclabsp-model-oe})	    &   $73.78$           												&   $61.69$          														&   $67.74$ 														&   $0.269$            												\\ 
		MPC Experimental \acrshort{armax} (\cref{tab:tclabsp-model-armax})	&   $91.79$           												&   $57.31$          														&   $74.55$ 														&   $0.370$            												\\ 
		PID (\cref{tab:pid_values})	                                        &   $84.97$           												&   $88.90$          														&   $86.94$ 														&   $0.554$            												\\ 
		MPC Experimental \acrshort{tf} (\cref{tab:tclabsp-model-tf})		&   $90.39$           												&   $84.58$          														&   $87.49$ 														&   $0.562$            												\\ 
		MPC Experimental \acrshort{arx}	(\cref{tab:tclabsp-model-arx})		&   $105.79$           												&   $113.15$          														&   $109.47$														&   $0.887$             											\\ 
		MPC Experimental \acrshort{ss} (\cref{eq:tclabsp-model-ss})			&   $136.28$           												&   $97.80$          														&   $117.04$														&   $1.000$             											\\ \bottomrule 
	\end{tabular}
	\caption*{Fonte: Autor}
\end{table}

\begin{table}[!h]
	\centering
	\caption{Performance dos controladores \acrshort{mpc} e \acrshort{pid} - ITSE}
	\label{tab:resultados_mpc_e_pid_manual_itse}
	\begin{tabular}{l|cc|c|c} \toprule
		{Modelo utilizado no controlador}              			            &	\begin{tabular}[c]{@{}c@{}}Sensor 1\\(x$10^7$)\end{tabular}		&	\begin{tabular}[c]{@{}c@{}}Sensor 2\\(x$10^7$)\end{tabular}     		& \begin{tabular}[c]{@{}c@{}}Média\\(x$10^7$)\end{tabular}		& 	\begin{tabular}[c]{@{}c@{}}Média\\normalizada\end{tabular}		\\ \midrule
		MPC Teórico \acrshort{ss} (\cref{eq:tclab_modelo_teorico})	        &   $74.01$           												&   $26.80$          														&   $50.41$														&   $0.000$             											\\ 
		\textbf{MPC Exp. \acrshort{tf} (Empírico)}                  		&   $\mathbf{75.15}$          										&   $\mathbf{32.08}$          												&   $\mathbf{53.61}$											&   $\mathbf{0.099}$												\\ 
		MPC Experimental \acrshort{oe} (\cref{tab:tclabsp-model-oe})	    &   $83.09$           												&   $33.31$          														&   $58.20$														&   $0.241$             											\\ 
		MPC Experimental \acrshort{armax} (\cref{tab:tclabsp-model-armax})	&   $96.70$           												&   $30.77$          														&   $63.74$														&   $0.413$             											\\ 
		MPC Experimental \acrshort{tf} (\cref{tab:tclabsp-model-tf})		&   $96.51$           												&   $45.02$          														&   $70.76$														&   $0.631$             											\\ 
		PID (\cref{tab:pid_values})	                                        &   $92.25$           												&   $55.69$          														&   $73.97$														&   $0.730$             											\\ 
		MPC Experimental \acrshort{ss} (\cref{eq:tclabsp-model-ss})			&   $114.29$           												&   $50.85$          														&   $82.57$														&   $0.997$             											\\ 
		MPC Experimental \acrshort{arx}	(\cref{tab:tclabsp-model-arx})		&   $96.37$           												&   $68.92$          														&   $82.65$														&   $1.000$             											\\ \bottomrule 
	\end{tabular}
	\caption*{Fonte: Autor}
\end{table}

\clearpage

\section{Sintonia}

Visando sintonizar empiricamente o controlador em questão, os seguintes passos foram seguidos:

\begin{itemize}
	\item Ajuste do cenário no \textit{MPC Design} para 300s, de modo a observar o sistema até seu regime estacionário;
	\item Ajuste dos limitantes (\textit{constraints}) para que os aquecedores não recebessem sinal maior que o permitido;
	\item Diminuição do tempo de amostragem para aumentar a velocidade entre iterações;
	\item Aumento gradativo do Horizonte de Predição até que nenhuma alteração perceptível fosse observada nos gráficos de resposta;
	\item Aumento gradativo do Horizonte de Controle até que nenhuma alteração perceptível fosse observada nos gráficos de resposta;
	\item Ajustar os seletores de "Robustez/Agressividade" e "Velocidade de estimativa de estados" utilizando \textit{Backtracking} até obter uma performance satisfatória.
\end{itemize}



\end{apendicesenv}