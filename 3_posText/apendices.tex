\begin{apendicesenv}

% Imprime uma página indicando o início dos apêndices
\partapendices

\chapter{Códigos-fonte}
\label{ch:codigos_extras}

\section{Método de pesquisa em grade}

\lstinputlisting[	
	caption={Pesquisa em grade com número escalar},
	captionpos=t,
	label={lst:grid_search_scalar},
	language=Python,
	style=Python_lang]
	{./4_Codes/grid_search_scalar.py}
	\begin{center}
		\makebox[\width]{Fonte: Autor, adaptado de \citeonline{Haugen2018}}
	\end{center}

\section{Método de pesquisa em grade com duas variáveis}

\lstinputlisting[	
	caption={Pesquisa em grade com duas variáveis},
	captionpos=t,
	label={lst:grid_search_vectorial},
	language=Python,
	style=Python_lang]
	{./4_Codes/grid_search_vectorial.py}
	\begin{center}
		\makebox[\width]{Fonte: Autor, adaptado de \citeonline{Haugen2018}}
	\end{center}

\section{Exemplo do método \acrlong{mhe}}

% (Brincalepe): O código da página 44 deve ser deslocado para o apêndice.
% (cont.) Se quiser, destaque apenas alguma parte mais relevante e mantenha no corpo do texto.
\lstinputlisting[	
	caption={Exemplo do método \acrlong{mhe}},
	captionpos=t,
	label={lst:mhe_example},
	language=Python,
	style=Python_lang]
	{./4_Codes/mhe_example.py}
	\begin{center}
		\makebox[\width]{Fonte: Autor, adaptado de \citeonline{Haugen2018}}
	\end{center}

\section{Exemplo de aplicação do \acrlong{mpc}}

\lstinputlisting[	
	caption={Exemplo de aplicação do \acrlong{mpc}},
	captionpos=t,
	label={lst:mpc_example},
	language=Python,
	style=Python_lang]
	{./4_Codes/mpc_example.py}
	\begin{center}
		\makebox[\width]{Fonte: Autor, adaptado de \citeonline{Haugen2018}}
	\end{center}
% (Brincalepe): O código da página 58 também pode ser deslocado para o Apêndice.
% (cont.) Seria interessante descrevê-lo no corpo de texto através de um diagrama.

\end{apendicesenv}