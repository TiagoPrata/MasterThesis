\setlength{\absparsep}{18pt} % ajusta o espaçamento dos parágrafos do resumo
\begin{resumo}[Abstract]
  \begin{otherlanguage*}{english}
    \acrshort{mpc} is a control technique that has gained attention and refinement over the last 50 years
    and it is currently one of the most prominent techniques in predictive control, having numerous
    commercial applications since its implementation in multivariate chemical process control,
    to the construction of logics for the orientation of autonomous vehicles.
    
    This dissertation deals with the development of a model based predictive control, applied
    to a pilot plant, addressing the whole process of construction and development of it,
    aiming to provide a clear understanding of each of the stages.
    
    This project can be divided into 6 different parts: basic theory of predictive control;
    theories about \acrshort{mpc}; systems identification; construction of the predictive controller;
    application and analysis. The whole practical part of this project is developed in \acrshort{matlab}
    and / or \textit{Python}, and implemented in an easily accessible educational plan, aiming for
    easy replicability.

    \vspace{\onelineskip}

    \noindent 
    \textbf{Keywords}: model predictive control; predictive contro; control theory; system identification
  \end{otherlanguage*}
\end{resumo}