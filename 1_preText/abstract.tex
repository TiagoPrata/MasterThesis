\setlength{\absparsep}{18pt} % ajusta o espaçamento dos parágrafos do resumo
\begin{resumo}[Abstract]
  \begin{otherlanguage*}{english}
    \acrshort{mpc} is a control technique that has been gaining attention and refinement over the last 50 years
    and it is currently one of the most prominent techniques in predictive control, having numerous
    commercial applications since its implementation in multivariate chemical process control,
    to the construction of logics for the orientation of autonomous vehicles.
    Initially, a theoretical survey on optimization and model predictive control is presented and then the plant
    that is the object of study in this work and its theoretical and experimental models is introduced.
    The practical part of this work is developed mainly in \acrshort{matlab} with the addition of some \textit{Python} code,
    and implemented in the pilot plant, aiming at an easy replicability.
    Finally, a comparison is presented between different \acrshort{mpc} controllers using the theoretical and experimental
    models raised from the plant and the verification of their performance in comparison with a PID controller.

    \vspace{\onelineskip}

    \noindent 
    \textbf{Keywords}: model predictive control; predictive contro; control theory; system identification
  \end{otherlanguage*}
\end{resumo}