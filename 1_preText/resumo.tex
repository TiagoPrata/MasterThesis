\newacronym{mpc}{MPC}{Controle Preditivo Baseado em Modelo}
\newacronym{matlab}{MATLAB\textsuperscript{\tiny\textregistered}}{\textit{Matrix Laboratory}}

\setlength{\absparsep}{18pt} % ajusta o espaçamento dos parágrafos do resumo
\begin{resumo}
    O \acrshort{mpc} é uma técnica de controle que vem ganhando atenção
    ao longo dos últimos 50 anos e, atualmente, ela figura entre uma das técnicas
    com maior destaque no controle preditivo, tendo inúmeras aplicações comerciais,
    desde sua implementação no controle de processos multivariáveis em indústrias químicas,
    até a construção de lógicas para a orientação de veículos autônomos. Esta dissertação
    trata da implementação de um controle preditivo baseado em modelo,
    aplicado a uma planta piloto, abordando todo o processo de construção e desenvolvimento
    do mesmo, visando proporcionar um entendimento claro de cada um das etapas. Inicialmente
    é apresentado um levantamento teórico sobre otimização e controle preditivo baseado em
    modelo e então introduzida a planta objeto de estudo desse trabalho e suas modelagens teóricas e
    experimentais. A parte prática deste trabalho é desenvolvida principalmente em \acrshort{matlab}
    com o complemento de alguns códigos em \textit{Python}, e implementada na planta piloto, visando
    uma fácil replicabilidade. Por fim é apresentado um comparativo entre diferentes controladores MPC
    utilizando os modelos teóricos e experimentais levantados da planta e a verificação do
    seu desempenho em comparação com um controlador PID.
    
    \vspace{\onelineskip}

    \noindent 
    \textbf{Palavras-chave}: controle preditivo baseado em modelo; controle preditivo; teoria de controle; identificação de sistemas
\end{resumo}