\newacronym{mpc}{MPC}{Controle Preditivo Baseado em Modelo}
\newacronym{matlab}{MATLAB\textsuperscript{\tiny\textregistered}}{\textit{Matrix Laboratory}}

\setlength{\absparsep}{18pt} % ajusta o espaçamento dos parágrafos do resumo
\begin{resumo}
    O \acrshort{mpc} é uma técnica de controle que vem ganhando atenção e refinamento
    ao longo dos últimos 50 anos e, atualmente, ela figura entre uma das técnicas
    com maior destaque no controle preditivo, tendo inúmeras aplicações comerciais,
    desde sua implementação no controle de processos multivariáveis em indústrias químicas,
    até a construção de lógicas para a orientação de veículos autônomos.

    Esta dissertação trata do desenvolvimento de um controle preditivo baseado em modelo,
    aplicado a uma planta piloto, abordando todo o processo de construção e desenvolvimento
    do mesmo, visando proporcionar um entendimento claro de cada um das etapas.

    É possível dividir este projeto em 6 diferentes partes: teoria base do controle
    preditivo; teoria sobre o \acrshort{mpc}; identificação de sistemas; construção do
    controlador preditivo; aplicação e análises. Toda a parte prática deste trabalho é
    desenvolvida em \acrshort{matlab} e/ou \textit{Python}, e implementada em uma
    planta educacional de fácil acesso, visando uma fácil replicabilidade.
    
    \vspace{\onelineskip}

    \noindent 
    \textbf{Palavras-chave}: controle preditivo baseado em modelo; controle preditivo; teoria de controle; identificação de sistemas
\end{resumo}