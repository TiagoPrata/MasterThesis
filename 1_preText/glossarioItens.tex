%% ---
%% entradas do glossario
%% ---
% \newglossaryentry{pai}{
%                name={pai},
%                plural={pai},
%                description={este é uma entrada pai, que possui outras
%                subentradas.} }
%
% \newglossaryentry{componente}{
%                name={componente},
%                plural={componentes},
%                parent=pai,
%                description={descriação da entrada componente.} }
% 
% \newglossaryentry{filho}{
%                name={filho},
%                plural={filhos},
%                parent=pai,
%                description={isto é uma entrada filha da entrada de nome
%                \gls{pai}. Trata-se de uma entrada irmã da entrada
%                \gls{componente}.} }
% 
%\newglossaryentry{equilibrio}{
%                name={equilíbrio da configuração},
%                see=[veja também]{componente},
%                description={consistência entre os \glspl{componente}}
%                }
%
%\newglossaryentry{latex}{
%                name={LaTeX},
%                description={ferramenta de computador para autoria de
%                documentos criada por D. E. Knuth} }
%
%\newglossaryentry{abntex2}{
%                name={abnTeX2},
%                see=latex,
%                description={suíte para LaTeX que atende os requisitos das
%                normas da ABNT para elaboração de documentos técnicos e científicos brasileiros} }
%% ---